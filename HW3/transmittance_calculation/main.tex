\documentclass[11pt,addpoints,answers]{exam}
\usepackage[margin=1in]{geometry}
\usepackage{graphicx}
\usepackage[svgname]{xcolor}
\usepackage{url}
\usepackage{datetime}
\usepackage{color}
\usepackage[many]{tcolorbox}
\usepackage{hyperref}

\newcommand{\courseNum}{\href{https://learning3d.github.io/index.html}{16825}}
\newcommand{\courseName}{\href{https://learning3d.github.io/index.html}{Learning for 3D Vision}}
\newcommand{\courseSem}{\href{https://learning3d.github.io/index.html}{Spring 2024}}
\newcommand{\courseUrl}{\url{https://piazza.com/cmu/spring2024/16825}}
\newcommand{\hwNum}{Problem Set 3}
\newcommand{\hwTopic}{Volume Rendering}
\newcommand{\hwName}{\hwNum: \hwTopic}
\newcommand{\outDate}{Feb. 21, 2024}
\newcommand{\dueDate}{Mar. 13, 2024 11:59 PM}
\newcommand{\instructorName}{Shubham Tulsiani}
\newcommand{\taNames}{Anurag Ghosh, Ayush Jain, Bharath Raj, Ruihan Gao, Shun Iwase}

\lhead{\hwName}
\rhead{\courseNum}
\cfoot{\thepage{} of \numpages{}}

\title{\textsc{\hwName}} % Title


\author{}

\date{}


%%%%%%%%%%%%%%%%%%%%%%%%%%
% Document configuration %
%%%%%%%%%%%%%%%%%%%%%%%%%%

% Don't display a date in the title and remove the white space
% \predate{}
% \postdate{}
% \date{}

%%%%%%%%%%%%%%%%%%
% Begin Document %
%%%%%%%%%%%%%%%%%%


\begin{document}

\section*{}
\begin{center}
  \textsc{\LARGE \hwNum} \\
  \vspace{1em}
  \textsc{\large \courseNum{} \courseName{} (\courseSem)} \\
  \courseUrl\\
  \vspace{1em}
  OUT: \outDate \\
  DUE: \dueDate \\
  Instructor: \instructorName \\
  TAs: \taNames
\end{center}


% Default to visible (but empty) solution box.
\newtcolorbox[]{studentsolution}[1][]{%
    breakable,
    enhanced,
    colback=white,
    title=Solution,
    #1
}

\begin{questions}
\question \textbf{[10 pts]}
\begin{figure}[h]
    \centering
    \includegraphics[width=\textwidth]{./figure1.png}
    \caption{A ray through a non-homogeneous medium. The medium is composed of 3 segments ($y1y2$, $y2y3$, $y3y4$). Each segment has a different absorption coefficient, shown as $\sigma_1, \sigma_2, \sigma_3$ in the figure. The length of each segment is also annotated in the figure (1m means 1 meter).}
    \label{fig:q1}
\end{figure}

As shown in Figure~\ref{fig:q1}, we observe a ray going through a non-homogeneous medium. 
Please compute the following transmittance:
\begin{itemize}
    \item $T(y1, y2)$
    \item  $T(y2, y4)$
    \item $T(x, y4)$
    \item $T(x, y3)$
\end{itemize} 



\begin{tcolorbox}[fit,height=20cm, width=\textwidth, blank, borderline={0.5pt}{-2pt},halign=left, valign=center, nobeforeafter]


\begin{studentsolution}

We know,
\begin{equation}
    T(a, b) = e^{- \sigma(y) dy }
\end{equation}
Therefore we can find transmittance of the 3 segments as,
\begin{equation}
    T(y_1, y_2) = e^{- \sigma_1 * 2} = e^{-2}
\end{equation}
\begin{equation}
    T(y_2, y_3) = e^{- \sigma_2 * 1} = e^{-0.5}
\end{equation}
\begin{equation}
    T(y_3, y_4) = e^{- \sigma_3 * 3} = e^{-30}
\end{equation}

We also assume that there is a vacuum in the segment of x to y1, therefore,
\begin{equation}
    T(x, y1) = 1
\end{equation}

Now finding answers using multiplicative property of transmittance- 

1) $T(y_1, y_2)$

\begin{equation}
    T (y_1, y_2) = e^{-2} = 0.135
\end{equation}

2) $T(y_2, y_4)$

\begin{equation}
    T (y_2, y_4) = T(y_2. y_3) \times T (y_3, y_4) = e^{-0.5} \times e^{-30} = e^{-30.5} = 5.68 \times 10^{-14}
\end{equation}

3) $T(x, y_4)$

\begin{equation}
    T (X, y_4) = T(x. y_1) \times T(y_1. y_2) \times T(y_2. y_3) \times T (y_3, y_4) = 1 \times e^{-2} \times e^{-0.5} \times e^{-30} = e^{-32.5} = 7.68 \times 10^{-15}
\end{equation}

3) $T(x, y_3)$

\begin{equation}
    T (x, y_4) = T(x. y_1) \times T(y_1. y_2) \times T(y_2. y_3) = 1 \times e^{-2} \times e^{-0.5} = e^{-2.5} = 0.082
\end{equation}



\end{studentsolution}

\end{tcolorbox}
\end{questions}
\end{document}
